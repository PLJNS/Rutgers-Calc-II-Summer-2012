\documentclass[11pt]{article}

\usepackage{fullpage}
\usepackage{multirow}
\usepackage{mathtools}
\usepackage{graphicx}
\usepackage{tabularx}
\usepackage{enumerate}
\usepackage{amsmath, amssymb, graphics, setspace}

\newcommand{\mathsym}[1]{{}}
\newcommand{\unicode}[1]{{}}

\title{Workshop XI}

\author{Paul Jones \\
		Instructor Mariya Naumova\\
		Calculus II (01:640:152, section C2)}

\date{\today}

\usepackage{setspace}
\doublespacing

\begin{document}

\maketitle

\pagebreak

The thread length for a simple spool of cotton thread is 25 yards.
To celebrate Valentine's Day, purchase a spool of red thread and send it to your beloved with these instructions:

\begin{quotation}
Unwind the thread and arrange it in the shape of a cardioid, $r = A(1 - \sin\theta)$.
The area of that cardioid represents how much I love you compared to the ordinary Valentine's Day card!
\end{quotation}

Compute the arc length of $r = A(1 - \sin\theta)$ and find A so that the length is 25 yards. 
Then compute the area inside that cardioid. Sketch the result.

\begin{enumerate}[a)]

	\item Arc length
	
		\begin{itemize}
		
			\item The formula for arc length in polar coordinate:
			
				\begin{equation*}
					L = \int_a^b \sqrt{ \left[r(\theta)\right]^2 + 
					\left[ {{ dr(\theta) } \over { d\theta }} \right] ^2 } d\theta
				\end{equation*}
			
			\item Consider $\left[r(\theta)\right]^2$:
			
				\begin{equation*}
					r^2 = A^2(1 - \sin\theta)^2 = A^2(1 - \sin^2\theta - 2\sin\theta)
				\end{equation*}
				
			\item Consider $\left[ {{ dr(\theta) } \over { d\theta }} \right] ^2 $:
			
				\begin{equation*}
					r' = -A\cos\theta, \; (r')^2 = A^2\cos^2\theta
				\end{equation*}
				
			\item Based on my rough sketch, function is symmetrical after $[-\frac{\pi}{2}, \frac{\pi}{2}]$. 
			Plug into formula:
			
				\begin{equation*}
					L = 2\int_{-\frac{\pi}{2}}^{\frac{\pi}{2}} \sqrt{ A^2(1 - \sin^2\theta - 2\sin\theta) + A^2\cos^2\theta } \; d\theta
				\end{equation*}
				
			\item Simplify and evalutate:
			
				\begin{equation*}
					L = 2\sqrt{2}A \int_{-\frac{\pi}{2}}^{\frac{\pi}{2}} \sqrt{1 - \sin\theta} \; d\theta = 8A
				\end{equation*}
						
		\end{itemize}
	
	\item Surface area
	
		\begin{itemize}
		
			\item If we want an $A$ such that $L = 25$, $A$ must equal $\frac{25}{8}$. The equation for $SA$:
			
				\begin{equation*}
					SA = \frac12\int_a^b \left[r(\theta)\right]^2\, d\theta
				\end{equation*}
		
			\item Plugin for our values:
			
				\begin{equation*}
					SA = \left[\frac{25}{8}\right]^2\int_{-\frac\pi2}^{\frac\pi2} (1 - \sin\theta)^2 \, d\theta
				\end{equation*}
				
			\item Integrate and simplify:
				
				\begin{equation*}
					SA = \left[\frac{25}{8}\right]^2 \cdot \frac{3\pi}{2} \approx 46
				\end{equation*}
		\end{itemize}

	\item Sketch

\begin{figure}[h]
     \centering
          \label{graph.png}
          \includegraphics*[width=\linewidth]{graph}
\end{figure}

\end{enumerate}

\end{document}

