\documentclass[11pt]{article}

\usepackage{fullpage}
\usepackage{multirow}
\usepackage{mathtools}
\usepackage{graphicx}
\usepackage{tabularx}
\usepackage{enumerate}
\usepackage{amsmath, amssymb, graphics, setspace}

\newcommand{\mathsym}[1]{{}}
\newcommand{\unicode}[1]{{}}

\title{Workshop X}

\author{Paul Jones \\
		Instructor Mariya Naumova\\
		Calculus II (01:640:152, section C2)}

\date{\today}

\usepackage{setspace}
\doublespacing

\begin{document}

\maketitle

\pagebreak

\begin{enumerate}[a)]

	\item Consider an isosceles triangle $\bigtriangleup OPQ$ where $OP = OQ = R$
	and $\angle POQ = \theta$. Find $PQ$.
	
		\begin{itemize}
		
			\item If you were to split this isosceles triangle in half from
			$\theta$, one could say that:
			
			\begin{equation*}
				\sin\frac{\theta}{2} = \frac{\frac{1}{2}PQ}{R}
			\end{equation*}
			
			\item Solving for $PQ$ yields:

			\begin{equation*}
				PQ = \frac{\sin\frac{\theta}{2}R}{2}
			\end{equation*}			
			
		
		\end{itemize}
	
	\item Consider a circle of radius $R$. Suppose that a regular $n$-gon is 
	inscribed in this circle. Find the perimeter $P_n$ of this $n$-gon.
	
		\begin{itemize}
	
			\item In the previous exercise, the length of one side of this 
			$n$-gon was found. The perimeter is $n$ times the value of this
			operation.
			
			\item The perimeter is equal to one side of an $n$-polygon multiplied
			by $n$:
			
			\begin{equation*}
				2 n R \sin\frac{\pi}{n}
			\end{equation*}
		
		\end{itemize}
	
	\item Find the $\lim_{n \to \infty} P_n$.

		\begin{itemize}
		
			\item Factor out constants ($2R$), ``rearrange'', apply 
			L'H\^{o}pital's rule to indeterminate form $\frac{0}{0}$, 
			factor out $\pi$, and recognize that any multiple of $\pi$ 
			within cosine yields zero:
			
			\begin{equation*}
				2 R \lim_{n \to \infty} \frac{\sin\frac{\pi}{n}}{n^{-1}} = 
				2 \pi R \lim_{n \to \infty} \cos\pi t = 2 \pi R
			\end{equation*}
		
		\end{itemize}
	
	\item Using the arc-length integral, find the arc-length of a circle of radius $R$.
	Explain why this integral must give the same answer as the previous limit.

		\begin{itemize}
		
		\item When describing a circle, the $a$ and $b$ values must of 
			the integral must yield an entire circle verses simply a portion
			of an arc. In an attempt to keep things simple, one can pick the values
			of $[0, 2\pi]$.
			
			\item Being as a circle is not a function, there is no simple function
			to input into the arc-integral's formula. Any ``y as a function of x''
			will limit one to a portion of a circle, and a portion of an arc. 
			But if one were to parametrically define the circle with radius $R$ as
			$x(t) = Rcos(t)$ and $y(t) = Rsin(t)$ where $0 \le t \le 2\pi$, 
			then one would have a full circle.
			
			\item The arc-length integral for parametric equations, defined in 
			section 11.2, is as follows:
			
			\begin{equation*}
				s = \int_a^b \sqrt{x'(t)^2 + y'(t)^2}dt
			\end{equation*}
			
			\item Therefore the integral for the arc length of a circle with radius R is
			as follows:
			
			\begin{equation*}
			\int_0^{2\pi}\sqrt{(R \frac{d}{dt}\cos t)^2+(R \frac{d}{dt}\sin t)^2}\;dt
		=	\int_0^{2\pi}\sqrt{(-R \sin t)^2+(R \cos t)^2}\;dt
			\end{equation*}
			
			\begin{equation*}
			=	R \int_0^{2\pi}\sqrt{\sin^2 t + \cos^2 t}\;dt = 
			R \int_0^{2\pi} dt = 
			2 \pi R
			\end{equation*}
					
					
			\item The ``arc-length'' of a circle with radius $R$ should equal the 
			value of the perimeter of a polygon whose number of sides approach infinity because
			a circle's ``arc-length'' is its perimeter and a polygon with an infinite
			amount of sides is a circle. Every point on a circle is equidistant from the
			center, which could be thought as a polygon with an infinite amount of 
			infinitesimally small sides. A circle's ``arc-length'' is the perimeter of a circle because
			arc-length is usually some subdivision of some function, and in 
			the case of a circle that subdivision is the whole circle and the function
			is a circle.	
				
		\end{itemize}
		
\end{enumerate}

\end{document}

