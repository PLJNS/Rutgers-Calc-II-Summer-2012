\documentclass[11pt]{article}

\usepackage{fullpage}
\usepackage{multirow}
\usepackage{mathtools}
\usepackage{graphicx}
\usepackage{tabularx}
\usepackage{enumerate}
\usepackage{amsmath, amssymb, graphics, setspace}

\newcommand{\mathsym}[1]{{}}
\newcommand{\unicode}[1]{{}}

\title{Workshop V}

\author{Paul Jones \\
		Instructor Mariya Naumova\\
		Calculus II (01:640:152, section C2)}

\date{\today}

\usepackage{setspace}
\doublespacing

\begin{document}

\maketitle

\pagebreak

\begin{enumerate}[a)]

\item Compute $\int_1^2 \! \frac{\mathrm{d}x}{x}$

	\begin{itemize}
	
	\item Being as the power is ``negative,'' 
	add one to $x$ and ``flip the sign''
	
	\begin{equation*}
	 \left. -\frac{1}{x}\right|_1^2
	\end{equation*}
	
	\item Evaluate using Fundamental Theorem of Calculus:
	
	\begin{equation*}
	\frac{1}{2}
	\end{equation*}
	
	\end{itemize}
	
\item Compute $\int_1^2 \! \frac{\mathrm{d}x}{x(x - m)}$ if $m$ is a small positive number. What happens as $m$ approaches 0 from the right?

	\begin{itemize}
	
	\item What we are considering is:
	
	\begin{equation*}
	\lim_{m \to 0^{+}}  \int_1^2 \frac{\mathrm{d}x}{(x (x-m))}
	\end{equation*}
	
	\item Intuitively, it makes sense that if $m$ is getting really small and is multiplied by $x$, as it approaches zero from the right, it will become less and less significant and our answer will take the form of problem (a). 
	
	\item Consider the integral separately using partial fractions:
	
	\begin{equation*}
	\int_1^2 \frac{1}{mx - m^2}\mathrm{d}x - \int_1^2 \frac{1}{mx}\mathrm{d}x
	\end{equation*}

	\item Factor out constants:
	
	\begin{equation*}
	\frac{1}{m} \int_1^2 \frac{\mathrm{d}x}{x - m} - \frac{1}{m}\int_1^2 \frac{\mathrm{d}x}{x}
	\end{equation*}
	
	\item Integrate:
	
	\begin{equation*}
	\left.\frac{1}{m} \log(x - m)\right|_1^2 - \left.\frac{1}{m}\log(x)\right|_1^2
	\end{equation*}
	
	\item Now use the Fundamental Theorem of Calculus to evaluate, and bring back the limit from the beginning:
	
	\begin{equation*}
	\lim_{m\to 0^+} \, \left(\frac{\log (2-m)-\log (2)}{m}- \frac{\log (1-m)}{m}\right)
	\end{equation*}
	
	\item Evaluate the limit. Intuitively, consider that that $\frac{\log(2 - .001) - \log(2)}{.001}$ is approximately $-\frac{1}{2}$ and $\frac{\log(1 - .001)}{.001}$ is approximately negative one, our answer is, thankfully:
	
	\begin{equation*}
	\frac{1}{2}
	\end{equation*}
	
	\end{itemize}

\item Compute $\int_1^2 \! \frac{\mathrm{d}x}{x^2 + n}$ if $n$ is a small positive number. What happens as $n$ approaches 0 from the right?

	\begin{itemize} 
	
	\item Again, intuitively it makes a lot of sense that if the $n$ in the denominator becomes of less and less significance, then our integral becomes the same form as (a).
	
	\item The integral takes the form of the inverse tangent function. Consider:
	
	\begin{equation*}
	\int \frac{dx}{x^2 + a^2} = \frac{1}{a}\tan^{-1}\frac{u}{a} + C
	\end{equation*}
	
	\item Where $a = \sqrt{n}$:
	\begin{equation*}
	\int_1^2 \! \frac{\mathrm{d}x}{x^2 + n} = \left.\frac{\tan^{-1}\frac{x}{\sqrt{n}}}{\sqrt{n}}\right|_1^2
	\end{equation*}
	
	\item By Fundamental Theorem of Calculus, and insert the pertinent limit:
	
	\begin{equation*}
	\lim_{n \to 0^+}\left(\frac{\tan^{-1}\frac{2}{\sqrt{n}}}{\sqrt{n}} - \frac{\tan^{-1}\frac{1}{\sqrt{n}}}{\sqrt{n}}\right)
	\end{equation*}
	
	\item I can't evaluate that. I'm going to try something else. Consider where $x = \sqrt n \tan\Theta$ and $\mathrm{d}x = \sqrt n \sec^2\theta\mathrm{d}\theta$
	
	\begin{equation*}
	\int_1^2 \! \frac{\mathrm{d}x}{x^2 + n} = \int_1^2 \! \frac{\sqrt n\sec^2\theta\mathrm{d}\theta}{\sqrt n \tan^2\theta + n}
	\end{equation*}
	
	\end{itemize}

\item Sketch the graphs:

\begin{figure}[h]
     \centering
          \label{graph.jpg}
          \includegraphics*[height=.65\linewidth]{graph.jpg}
\end{figure}

\end{enumerate}


\end{document}