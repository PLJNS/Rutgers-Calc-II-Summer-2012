\documentclass[11pt]{article}

\usepackage{fullpage}
\usepackage{multirow}
\usepackage{mathtools}
\usepackage{graphicx}
\usepackage{tabularx}
\usepackage{enumerate}
\usepackage{amsmath, amssymb, graphics, setspace}

\newcommand{\mathsym}[1]{{}}
\newcommand{\unicode}[1]{{}}

\title{Workshop VII}

\author{Paul Jones \\
		Instructor Mariya Naumova\\
		Calculus II (01:640:152, section C2)}

\date{\today}

\usepackage{setspace}
\doublespacing

\begin{document}

\maketitle

\pagebreak

Consider an infinite series of the form:

\begin{equation*}
\pm 3 \pm 1 \pm \frac{1}{3} \pm \frac{1}{9} \pm \frac{1}{27} \pm ... \pm \frac{1}{3^n} \pm ...
\end{equation*}

The numbers 3, 1, $\frac{1}{3}$ etc., are given but \emph{you} will decide what the signs should be.

\begin{enumerate}[a)]

\item Can you choose the signs to make the series diverge?

\begin{itemize}

	\item Consider the value of the series if every single value is added to every other value.
	As we shall see, if one were to add up everything after $\frac{1}{3}$, the value is $\frac{1}{2}$.
	Added to the three and the one in the beginning of the series, this yields a value of 4.5.
	Similarly, if one were to subtract every value, the result would be -4.5.
	
	\item This means that no matter how one selects the signs, the value yielded will be between -4.5 and 4.5,
	making it impossible to diverge.

\end{itemize}

\item Can you choose the signs to make the series sum to 3.5?

\begin{itemize}

	\item Intuitively, if the desired sum is 3.5, at least the first term must be positive.
	So consider separately:
	
	\begin{equation*}
	\left.\frac{1}{3} + \frac{1}{9} + \frac{1}{27} + ... + \frac{1}{3^n}\right._{n \to \infty}
	\end{equation*}
	
	\item This is a geometric series, where $r = \frac{1}{3}$:
	
	\begin{equation*}
	\sum_{n = 1}^{\infty} \frac{1}{3^n} = \frac{\frac{1}{3}}{1 - \frac{1}{3}} = \frac{1}{3} \cdot \frac{2}{3} = \frac{1}{2}
	\end{equation*}
	
	\item Now, consider the question:
	
	\begin{equation*}
	\Box \; 3 \; \Box \; 1 \; \Box \; \sum_{n = 1}^{\infty} \frac{1}{3^n} = \Box \; 3 \; \Box \; 1 \; \Box \; \frac{1}{2}
	\end{equation*}
	
	\item If one adds one to three and then subtracts one half, the operation yield 3.5.
	
\end{itemize}

\clearpage

\item Can you choose the signs to make the series sum to 2.25?

\begin{itemize}

	\item Imagine what it would be like if one were able to pick only odd indices of the sum in the above problem, and what the sum would yield.
	It would look something like:
	
	\begin{equation*}
	\frac{1}{3^1} \pm \frac{1}{3^3} \pm \frac{1}{3^5} \pm ... = \frac{1}{3} \pm \frac{1}{27} \pm \frac{1}{243} \pm ... = 
	\frac{1}{3} \left(\frac{1}{9^0} \pm \frac{1}{9^1} \pm \frac{1}{9^2} \pm ... \right)
	\end{equation*}

	\item Consider separately:
	
	\begin{equation*}
	\frac{1}{3} \sum_{n = 0}^{\infty} \frac{1}{9^n} = \frac{1}{3} \cdot \frac{1}{1 - \frac{1}{9}} = \frac{3}{9} \cdot \frac{9}{8} = \frac{3}{8}
	\end{equation*}
	
	\item This is close to what I'm looking for, but not quite. 
	Now imagine what it would be like if one were able to pick only even indices of the sum in the same above problem.
	It would be something like:
	
	\begin{equation*}
	\frac{1}{3^2} \pm \frac{1}{3^4} \pm \frac{1}{3^6} \pm ... = \frac{1}{9} \pm \frac{1}{81} \pm \frac{1}{729} \pm ... = 
	\frac{1}{9^1} \pm \frac{1}{9^2} \pm \frac{1}{9^3} \pm ...
	\end{equation*}
	
	\item Consider separately:
	
	\begin{equation*}
	\sum_{n = 1}^{\infty} \frac{1}{9^n} = \frac{\frac{1}{9}}{1 - \frac{1}{9}} = \frac{1}{9} \cdot \frac{9}{8} = \frac{1}{8}
	\end{equation*}
	
	\item Now consider the question:
	
	\begin{equation*}
	\Box \; 3 \; \Box \; 1 \; \Box \; \frac{1}{3} \sum_{n = 0}^{\infty} \frac{1}{9^n} \; \Box \sum_{n = 1}^{\infty} \frac{1}{9^n} = 
	\Box \; 3 \; \Box \; 1 \; \Box \; \frac{3}{8} \; \Box \frac{1}{8}
	\end{equation*}
	
	\item The answer: 
	
	\begin{equation*}
	3 - 1 + \frac{3}{8} - \frac{1}{8} = 2.25
	\end{equation*}
	
	\item This means that if you add every odd power and subtract every even power of the original series, it would yield .25.
	
\end{itemize}

\end{enumerate}

\end{document}