\documentclass[11pt]{article}

\usepackage{fullpage}
\usepackage{multirow}
\usepackage{mathtools}
\usepackage{graphicx}
\usepackage{tabularx}
\usepackage{enumerate}
\usepackage{amsmath, amssymb, graphics, setspace}

\newcommand{\mathsym}[1]{{}}
\newcommand{\unicode}[1]{{}}

\title{Workshop VI}

\author{Paul Jones \\
		Instructor Mariya Naumova\\
		Calculus II (01:640:152, section C2)}

\date{\today}

\usepackage{setspace}
\doublespacing

\begin{document}

\maketitle

\pagebreak

\begin{itemize}

\item Two students are sharing a loaf of bread. Student A, now hungrier and more
ferocious, eats two thirds of the the loaf, then student B eats half of what
remains, then A eats two-thirds of what remains, then B eats half of what remains,
and so on. How much of the loaf will each student eat?

\begin{itemize}

	\item Begin working out the pattern:

	\begin{table}[h]
		\begin{center}
			\caption{Beginning the pattern}
				\begin{tabular}{|c c|c c|}
				\hline
				\multicolumn{2}{|c|}{Student A} & \multicolumn{2}{|c|}{Student B} \\
				Eats & Remains & Eats & Remains \\\hline
				$\frac{2}{3}$ & $\frac{1}{3}$ & $\frac{1}{6}$ & $\frac{1}{6}$ \\\hline
				$\frac{1}{9}$ & $\frac{1}{18}$ & $\frac{1}{36}$ & $\frac{1}{36}$ \\\hline
				$\frac{1}{9 \cdot 2 \cdot 3}$ & $\frac{1}{18 \cdot 2 \cdot 3}$ & $\frac{1}{18 \cdot 2 \cdot 3 \cdot 2}$ & $\frac{1}{18 \cdot 2 \cdot 3 \cdot 2}$ \\
				\hline
				\end{tabular}
			\label{ }
		\end{center}
	\end{table}
	
	\item Consider Student B: She consumes bread according to this sum:
	
	\begin{equation*}
	\sum_{i=1}^{\infty}\left(\frac{1}{6}\right)^i
	\end{equation*}
	
	\item Recall the geometric series:
	
	\begin{equation*}
	\sum_{i=1}^{\infty}\left(\frac{1}{6}\right)^i = \frac{1}{1 - \frac{1}{6}} - 1 = \frac{1}{5}
	\end{equation*}
	
	\item If Student B consumes this much, the other student must consume:
	
	\begin{equation*}
	\frac{4}{5}
	\end{equation*}
	
\end{itemize}

\item For each sequence, stat exactly how large $n$ must be to ensure that the term $a_n$ of the sequence satisfies $|a_n| < 10^{-4}$. Then, use the information to explain which sequences approaches zero most rapidly and which approaches zero least rapidly.

\begin{itemize}

	\item Consider the sequence:
	
	\begin{equation*}
	\left\{ \frac{1}{\sqrt{n}} \right\}_{n = 1}^{\infty}
	\end{equation*}
	
	\item Now consider the definition of the limit of sequence:
	
	\begin{equation*}
	\forall\epsilon>0 \; \exists n_\epsilon \in N \; \forall n \ge n_\epsilon : \left|a_n - b\right| < \epsilon
	\end{equation*}
	
	\item The limit of the sequence is zero. Therefore:
	
	\begin{equation*}
	\left|\frac{1}{\sqrt{n}} - 0\right| < \epsilon
	\end{equation*}
	
	\item And therefore:
	
	\begin{equation*}
	-\frac{1}{\sqrt{n}} < \epsilon \therefore n > 0
	\end{equation*}
	
	\item And:
	
	\begin{equation*}
	\frac{1}{\sqrt{n}} < \epsilon \therefore \left(\frac{1}{\epsilon}\right)^2 > n
	\end{equation*}
	
	\item a) Thus, if the index is greater than $10^8$, the result will be smaller than $10^{-4}$.
	
	\item b) Thus, if the index is greater than $10^{16}$, the result will be smaller than $10^{-8}$.
	
	\item Consider the sequence:
	
	\begin{equation*}
	\left\{ \frac{1}{10^n} \right\}_{n = 1}^{\infty}
	\end{equation*}
	
	\item Again, by definition of the limit of a sequence (our limit is zero):
	
	\begin{equation*}
	\left|\frac{1}{10^n}\right| < \epsilon
	\end{equation*}
	
	\item Therefore, there two cases are:
	
	\begin{equation*}
	-\frac{1}{10^n} < \epsilon \therefore n > 0
	\end{equation*}
	
	\item And:
	
	\begin{equation*}
	\frac{1}{\sqrt{n}} < \epsilon \therefore 4\log10 < n \therefore 4 < n
	\end{equation*}
	
	\item a) Thus, if the index is greater than 4, the result will be smaller than $10^{-4}$.
	
	\item b) Similarly, if the index is greater than 8, the result will be smaller than $10^{-8}$.
	
	\item c) Being as the index 4 brings the second sequence closer to a small number than the index $10^8$ brings our first sequence to the same small number, the second sequence must approach zero faster
	
\end{itemize}

\end{itemize}


\end{document}