\documentclass[11pt]{article}

\usepackage{fullpage}
\usepackage{multirow}
\usepackage{mathtools}
\usepackage{graphicx}
\usepackage{tabularx}
\title{Workshop III}

\author{Paul Jones \\
		Instructor Mariya Naumova\\
		Calculus II (01:640:152, section C2)}

\date{\today}

\usepackage{setspace}
\doublespacing

\begin{document}


\maketitle

\pagebreak

\begin{enumerate}

\item Suppose $m$ and $n$ integers. Compute $ \int_0^{2\pi} \! cos(mx)cos(nx) \, \mathrm{d} x$.

\begin{itemize}

\item Consider the trigonometric identity:

\begin{equation*}
\frac{cos(a + b) + cos(a - b)}{2} = cos(a)cos(b)
\end{equation*}

\item Plug in $a = mx$ and $b = nx$:

\begin{equation*}
\int_0^{2\pi} \! \cos(mx)\cos(nx) \, \mathrm{d} x = \frac{1}{2}\int_0^{2\pi} \! \cos(mx + nx) + \cos(mx - nx) \, \mathrm{d} x
\end{equation*}

\item This integrates to:

\begin{equation*}
\left. \frac{1}{2}\left(\ \frac{\sin(mx+nx)}{m + n} + \frac{\sin(mx - nx)}{m - n} \right)\ \right|_0^{2\pi}
\end{equation*}

\item Evaluate between 0 and $2\pi$:

\begin{equation*}
\frac{1}{2}\left(\ \frac{\sin(2\pi(m+n))}{m + n} + \frac{\sin(2\pi(m - n))}{m - n} \right)\ = 0
\end{equation*}

\item {\bf Solution}: Any multiple of $2\pi$ within $\sin$ will yield zero.

\item But $m = n$ and $m = -n$ are different cases. $m = n$:

\begin{equation*}
\int_0^{2\pi} \! \cos^2(mx) \, \mathrm{d} x = \frac{1}{2}\int_0^{2\pi} \! 1 + \cos(2mx) \, \mathrm{d} x = \left. \left(\ x + \frac{\sin(2mx)}{2m} \right)\ \right|_0^{2\pi} = \pi
\end{equation*}

\item $m = -n$:
\begin{equation*}
\int_0^{2\pi} \! \cos(-nx)\cos(nx) \, \mathrm{d} x = \frac{1}{2}\int_0^{2\pi} \! 1 + \cos(-2nx) \, \mathrm{d} x = \pi + \frac{\sin(4n\pi))}{4n} = \pi
\end{equation*}

\end{itemize}

\item Suppose $f(x) = A\cos(x) + B\cos(2x) + C\cos(3x)$ and you know that
\begin{table}[h]
\begin{center}
	\begin{tabularx}{1\textwidth}{X X X}
	$ \int_0^{2\pi} \! f(x)\cos(x) \, \mathrm{d} x = 5 $ & 
	$ \int_0^{2\pi} \! f(x)\cos(2x) \, \mathrm{d} x = 6$ &
	$\int_0^{2\pi} \! f(x)\cos(3x) \, \mathrm{d} x = 7$ \\
	\end{tabularx}
\end{center}
\end{table}

\begin{itemize}

\item Substitute in $f(x)$ into each integral:

\begin{equation*}
\int_0^{2\pi} \! A\cos^2(x) + B\cos(x)\cos(2x) + C\cos(x)\cos(3x) \, \mathrm{d} x
\end{equation*}

\item These can be considered separately. 

\begin{equation*}
A \int_0^{2\pi} \! \cos^2(x) \, \mathrm{d} x + 
B \int_0^{2\pi} \! \cos(x)\cos(2x) \, \mathrm{d} x + 
C \int_0^{2\pi} \! \cos(x)\cos(3x) \, \mathrm{d} x = 5
\end{equation*}

\item By the proof in the previous section, the last two integrals equal zero, and the first equals pi, therefore:

\begin{equation*}
A\pi = 5, A = \frac{5}{\pi}
\end{equation*}

\item Now, substitute $f(x)$ into the next integral:

\begin{equation*}
A \int_0^{2\pi} \! \cos(x)\cos(2x) \, \mathrm{d} x + 
B \int_0^{2\pi} \! \cos^2(2x) \, \mathrm{d} x + 
C \int_0^{2\pi} \! \cos(2x)\cos(3x) \, \mathrm{d} x = 6
\end{equation*}

\item Using the same logic, the integrals multiplied by $A$ and $C$ are zero, and the integral beginning with $B$ equals pi, therefore:

\begin{equation*}
B\pi = 6, B = \frac{6}{\pi}
\end{equation*}

\item Finally, apply the same process to the last integral in the question:

\begin{equation*}
A \int_0^{2\pi} \! \cos(x)\cos(3x) \, \mathrm{d} x + 
B \int_0^{2\pi} \! \cos(3x)\cos(2x) \, \mathrm{d} x + 
C \int_0^{2\pi} \! \cos^2(3x) \, \mathrm{d} x = 7
\end{equation*}

\begin{equation*}
C\pi = 7, C = \frac{7}{\pi}
\end{equation*}


\end{itemize}

\end{enumerate}

\end{document}